\documentclass[12pt, oneside, a4paper,english,brazil]{abntex2}

\usepackage[nonumberlist=true]{glossaries}
\usepackage[utf8]{inputenc} % Codificação do documento (conversão automática
                            % dos acentos)
\usepackage{lmodern} % Usa a fonte Latin Modern
\usepackage[T1]{fontenc} % Seleção de códigos de fonte.
\usepackage{lastpage} % Usado pela Ficha catalográfica
\usepackage{indentfirst} % Indenta o primeiro parágrafo de cada seção.
\usepackage{color} % Controle das cores
\usepackage{graphicx,url}
\usepackage{microtype} % para melhorias de justificação
\usepackage{longtable}
\usepackage[super]{nth}
\usepackage[num]{abntex2cite}
\usepackage{hyperref}
\usepackage{amsfonts}
\usepackage{minted}
\usepackage{proof}
\usepackage{amsmath,amsfonts,amsthm,bm}
\usepackage{mathtools}
\usepackage{appendix}

\renewcommand{\appendixtocname}{Ap\^endices}
\renewcommand{\appendixpagename}{Ap\^endices}

\DeclarePairedDelimiter\floor{\lfloor}{\rfloor}
\DeclarePairedDelimiter\ceil{\lceil}{\rceil}
\DeclareUnicodeCharacter{2200}{$\mathbb{\forall}$}
\DeclareUnicodeCharacter{227C}{$\preceq$}
\DeclareUnicodeCharacter{03B1}{$\alpha$}
\DeclareUnicodeCharacter{2115}{$\mathbb{N}$}
\DeclareUnicodeCharacter{03BB}{$\lambda$}
\DeclareUnicodeCharacter{2081}{$_1$}
\DeclareUnicodeCharacter{2082}{$_2$}
\DeclareUnicodeCharacter{2264}{$\le$}

\titulo
  {Formalização da Complexidade Temporal de Algoritmos de Ordenação em Lean}
\autor{Tomaz Gomes Mascarenhas}
\local{Belo Horizonte, Minas Gerais}
\data{2021}
\orientador{Haniel Barbosa}
\instituicao{%
Universidade Federal de Minas Gerais
\par Instituto de Ciências Exatas
\par Departamento de Ciência da Computação}
\tipotrabalho{Projeto Orientado em Computação}

% \preambulo{Proposta de Pesquisa Científica}

\makeglossaries

\setglossarystyle{altlist}

\newglossaryentry{proj12}{name = $proj_{1,2}(p)$, description={Par contendo o primeiro e segundo elementos da tupla ordenada $p$}}

\newglossaryentry{HI}{name = HI, description = {Hip\'oese Indutiva}}

\newglossaryentry{notleq}{name = $a \not\le b$, description = {Nega\c{c}\~ao de $a \le b$}}

\newglossaryentry{length}{name = $length(l)$, description={N\'umero de elementos na lista $l$}}

\newglossaryentry{proj3}{name = $proj_{3}(p)$, description={Terceiro elemento da tupla ordenada $p$}}

\newglossaryentry{proj2}{name = $proj_{2}(p)$, description={Segundo elemento da tupla ordenada $p$}}

\newglossaryentry{proj1}{name = $proj_{1}(p)$, description={Primeiro elemento da tupla ordenada $p$}}

\newglossaryentry{listindex}{name = $[a]_{i}$, description={$i$-\'esimo elemento da lista $a$}}

\newglossaryentry{listempty}{name = $[]$, description = {Lista vazia}}

\newglossaryentry{listcons}{name = $h :: t$, description={Lista composta pelo elemento $h$ seguido pela lista $t$}}

\newglossaryentry{listtype}{name = $[A]$, description={Tipo das listas de elementos do tipo $A$}}

\newglossaryentry{typing}{name=$a : A$, description={Termo $a$ tem tipo $A$}}

\newglossaryentry{functype}{ name=$A \rightarrow B$, description={Tipo das fun\c{c}\~oes do tipo $A$ para o tipo $B$}  }

\newglossaryentry{implies}{
                name=$A \Rightarrow B$,
                description={Proposi\c{c}\~ao $A$ implica em proposi\c{c}\~ao $B$} }
\begin{document}

\imprimircapa
\imprimirfolhaderosto

\tableofcontents* %\sumario %comando que gera o sumário automaticamente

\glsaddall
\printglossaries

\textual

\chapter{Introdução}
\section{Hist\'oria}

\subsection*{Teoria dos Tipos}

\qquad Em 1901, o matemático Bertrand Russell descobriu uma contradição lógica nos
fundamentos da teoria dos conjuntos, a qual formava a base da matemática naquela época.
Essa descoberta veio da observação de que, como não existem limitações para o conteúdo
de um conjunto, nada o impede de conter a si mesmo. De fato, o conjunto de todas as
ideias é uma ideia, e deve estar contido em si mesmo, por exemplo. Desse modo, é possível
separar os conjuntos em duas categorias: aqueles que contém e aqueles que não contém a
si próprios. Formalmente, denotando por $U$ o conjunto de todos os conjuntos, teríamos:

\begin{center}
  $M = \{A\, |\, A\, \in A\}$ \\
  $N = \{A\, |\, A\, \notin A\}$ \\
  $U = M\, \cup\, N$
\end{center}

\qquad Dessa definição surge uma dúvida: $N$ contém a si próprio? Se ele não contiver,
então, sua definição implicaria que ele deveria se conter. Por outro lado, se $N \in N$, ele
estaria contrariando sua definição, já que $N$ só aceita conjuntos que não contém a si
mesmos. Em ambos os casos, temos uma contradição. Esse é um dos exemplos mais
famosos de impredicatividade, isto é, um objeto que referencia a si mesmo.

\qquad Como uma resposta para esse problema, Russell criou o campo que ficou conhecido
como Teoria dos Tipos. A ideia central era dividir os conjuntos em classes, que seriam
representadas por números naturais, e estabelecer regras que tornem impossível um
conjunto conter a si próprio. A classe de número $0$ representa todos os conjuntos que
não possuem conjuntos como elementos. A classe $1$ se referia a todos os conjuntos que,
se possuírem conjuntos como elementos, estes deverão pertencer a classe $0$. De modo
geral, conjuntos da classe $i$ só podem conter conjuntos de classes $j$, tal que $j < i$. Desse
modo, não podem existir conjuntos que contém a si próprios e o paradoxo deixa de existir.
Embora tenha resolvido esse problema, a teoria é menos expressiva do que sua antecessora.
Um exemplo disso \'e a impossibilidade de representar um tipo universal, que
contenha todos os outros tipos. Esse tipo de limita\c{c}\~ao fez com que a
teoria de Russell não fosse amplamente adotada pelos matemáticos.

\qquad Contudo, com o surgimento da Ciência da Computação, as ideias da Teoria dos Tipos
encontraram uma nova aplicação nas Linguagens de Programação. A maioria delas adota
algo que é conhecido como Sistema de Tipos. Nas palavras de Pierce \cite{pierce}, ``Um sistema de
tipos é um método sintático tratável para demonstrar a isenção de certos comportamentos
em um programa por meio da classificação de senten\c{c}as de acordo com as espécies de valores
que elas computam''. Por exemplo, sem um sistema desses, a expressão \textsf{``word'' + 2} poderia
ser considerada válida em uma linguagem, embora não esteja claro o que ela quer dizer.
Atribuindo ao objeto \textsf{``word''} o tipo String, ao objeto \textsf{2} o tipo Integer e ao operador
\textsf{+} a propriedade de que deve ser usado com objetos de mesmo tipo, por exemplo, é
possível identificar essa expressão como inválida estaticamente, isto \'e, sem a necessidade de tentar executar o programa e receber um erro (note a
semelhança entre essas restrições relacionadas a termos e as restrições estabelecidas por
Russell, relacionadas a conjuntos).

\qquad Embora esse exemplo possa parecer trivial, em softwares grandes e com relações
complexas entre classes podem surgir problemas muito mais sutis e que não se manifestariam
imediatamente na execução.

\subsection*{Isomorfismo de Curry-Howard}
\qquad Em meados do século XX, foi descoberto que os Sistemas de Tipos possuíam uma
relação muito profunda com a lógica. Considere, por exemplo, uma função $f : A \rightarrow B$
sendo $A$ e $B$ tipos quaisquer. Uma regra clássica dos sistemas de tipo é algo como:

\begin{center}
  $\infer{f(a) : B}{f : A \rightarrow B & a : A}$
\end{center}

\qquad Ou seja, se $f$ é uma função de $A$ para $B$ e $a$ é um objeto do tipo $A$, então aplicar
$f$ em $a$ produz algo do tipo $B$. Na lógica proposicional existe uma regra de inferência
semelhante a isso. Se $P$ e $Q$ são proposições e $P \Rightarrow Q$ denota que a proposição $P$ implica
em $Q$, a regra chamada Modus Ponens estabelece a seguinte relação:

\begin{center}
  $\infer{Q}{P \Rightarrow Q & P}$
\end{center}

\qquad Se possuímos uma demonstração de que a proposição $P$ é verdadeira e que $P$
implica em $Q$, então podemos usar essa regra para concluir que $Q$ é verdadeiro.

\qquad De fato, esses dois sistemas não são apenas semelhantes, mas equivalentes. Note
que a única diferença entre as duas regras é a natureza dos objetos aos quais elas se referem.
Se identificarmos proposições como tipos e demonstrações como termos de um determinado
tipo, então não existe diferença entre elas. Podemos interpretar a proposição $P \Rightarrow Q$ como
o tipo das funções que levam demonstrações de $P$ em demonstrações de $Q$.

\qquad Um isomorfismo entre duas categorias de objetos matemáticos é um mapeamento
entre objetos da primeira categoria e os da segunda, preservando suas propriedades.
Essa correspondência recebeu o nome Isomorfismo de Curry-Howard, em homenagem
aos pesquisadores que a descobriram. Cada elemento da lógica corresponde a alguma
construção nas linguagens de programação. Alguns exemplos são:
\begin{table}[H]
  \centering
    \begin{tabular}{cc}\toprule
      Versão em Lógica & Versão em Programação  \\\midrule
      Proposição & Tipo \\
      Demonstração & Termo do tipo correspondente \\
      Indução & Recursão \\
      Conjunção & Produto de tipos \\
      Disjunção & União de tipos \\
      Implicação & Função \\
      Negação & Função para um tipo vazio \\\bottomrule
    \end{tabular}
\end{table}

\subsection*{Assistentes de Demonstra\c{c}\~ao}
\qquad Essa equivalência contribuiu para a cria\c{c}\~ao dos \textbf{Assistentes de Demonstra\c{c}\~ao}. Estes
são sistemas que dão suporte a certas linguagens de programação, fazendo com que elas
concretizem essa correspondência e seja possível usá-las para demonstrar teoremas por
meio da escrita de programas. Tais demonstrações podem ser verificadas mecanicamente
pelo compilador da linguagem, fazendo com que o grau de confiabilidade que elas atinjam
seja muito superior ao de demonstrações convencionais. Existem assistentes de
demonstra\c{c}\~ao que
n\~ao usam o Isomorfismo de Curry-Howard, mas outros modelos para implementar
as mesmas funcionalidades. Um desses modelos \'e o \textit{Logic for Computable
Functions} \cite{lcf}.

\qquad De acordo com Geuvers \cite{geuvers} (p. 4): ``Uma demonstração matemática pode ser
reduzida a uma série de etapas muito breves que podem ser verificadas de forma sim-
ples e irrefutável''. Tais etapas são tão pequenas que quase sempre são resumidas em
demonstrações, e sua corretude fica subentendida. No entanto, é possível que esse resumo
omita alguma falha lógica, invalidando toda a demonstração de uma forma difícil de ser
detectada, como j\'a aconteceu diversas vezes na hist\'oria da matem\'atica.

\qquad Ao usar um assistente, não é possível omitir tais partes da demonstração dessa
maneira, tornando sua escrita mais trabalhosa. Contudo, ele verifica cada passo da dedução
e o valida, tornando impossível que erros lógicos sejam aceitos. Por causa do isomorfismo
de Curry-Howard, o compilador é capaz de tratar cada passo lógico como a aplicação de
uma função de um certo tipo, assim, caso exista uma inconsistência no argumento usado
ele será capaz de identificar uma inconsistência entre os tipos dos objetos, resultando em
um erro de compilação.

\section{Aplica\c{c}\~oes}

\qquad Talvez o caso mais famoso de aplicação dos assistentes seja a demonstração do
Teorema das Quatro Cores. Tal teorema afirma que, dado qualquer mapa plano, dividido
em regiões, sempre é possível pintá-lo com no máximo quatro cores, de modo que duas
regiões vizinhas nunca possuam a mesma cor. Esse fato foi conjecturado pelo matemático
Francis Guthrie em 1852 e permaneceu por mais de 100 anos sem ser demonstrado, até
que em 1976, sua veracidade foi verificada, com a ajuda de um computador IBM 360. A
demonstração consistia em reduzir todos os mapas possíveis em 1936 configurações, de
modo que para demonstrar o teorema bastaria verificar uma por uma delas e garantir que
nunca são necessárias mais do que quatro cores para pintá-las. A única possibilidade de
realizar tal tarefa foi por meio da ajuda de computadores, e, mesmo os mais eficazes da
época levaram mais de mil horas para isso. O uso desse tipo de técnica foi questionado por
parte da comunidade matemática, principalmente pelo fato de ser necessário acreditar que
os softwares utilizados para colorir e gerar os mapas estariam corretos. Anos depois, em
2005, a demonstração foi formalizada dentro de um assistente de demonstra\c{c}\~ao chamado Coq \cite{fourcolor}, fazendo com
que fosse necessário acreditar apenas que seu kernel estaria correto. Exitem divergências
quanto ao uso desse tipo de demonstração na matemática, mas a comunidade no geral
tem se mostrado confiante nessa ferramenta. Por exemplo, 29 dos 62 participantes de um evento
relacionado a um assistente chamado Lean \cite{2} se identificaram como matemáticos \cite{hitchhiker} (p. 8).

\qquad Um outro campo que também tem se beneficiado desse mecanismo é a Engenharia
de Software. Uma das propriedades mais desejáveis quando se desenvolve programas é a
corretude, ou seja, a garantia de que, para qualquer entrada que receba, o software irá
respeitar alguma especificação fornecida previamente. Usando linguagens com suporte a
assistentes, é possível escrever funções, demonstrar sua corretude formalmente e exportá-la
para uma outra linguagem mais usual, garantindo sua corretude. Em softwares de miss\~ao cr\'itica,
esse tipo de prática pode evitar prejuízos catastróficos.

\section{Contexto}

\qquad O objetivo deste trabalho \'e usar Lean, um dos assistentes de demonstra\c{c}\~ao
mais relevantes atualmente, para formalizar algum conceito que ainda n\~ao tenha sido
amplamente explorado no contexto dos assistentes e que seja relevante para a comunidade.
De acordo com membros ativos da comunidade  que
utiliza
Lean\footnote{\url{https://leanprover.zulipchat.com/\#narrow/stream/113488-general/topic/BSc.20Final.20Project}}, a complexidade temporal de algoritmos é um conceito relevante que
ainda não foi formalizado em nenhum módulo de sua biblioteca de formaliza\c{c}\~oes.
Em particular, a complexidade temporal de algoritmos básicos de ordenação, que poderia ser
formalizada sem usar técnicas muito avançadas. Assim, o escopo do trabalho será
a demonstração de um limite, com base no tamanho da entrada, para o número de
operações feitas pelos algoritmos Insertion Sort e Merge Sort, implementados na
biblioteca do
Lean\footnote{\url{https://github.com/leanprover-community/mathlib/blob/master/src/data/list/sort.lean}}.

\qquad Na se\c{c}\~ao 2, apresentamos os conceitos b\'asicos que s\~ao abordados por este trabalho.
Na se\c{c}\~ao 3, definimos matematicamente
os algoritmos que ser\~ao formalizados. Em seguida, na se\c{c}\~ao 4 apresentamos
demonstra\c{c}\~oes convencionais dos principais teoremas que est\~ao no escopo
do trabalho, usando as defini\c{c}\~oes da se\c{c}\~ao anterior. Finalmente,
na se\c{c}\~ao 5, mostramos a implementa\c{c}\~ao em Lean dos algoritmos e dos teoremas,
junto com as principais ideias necess\'arias para escrever suas demonstra\c{c}\~oes
no assistente.

\chapter{Conceitos B\'asicos}
\section{Algoritmos de Ordena\c{c}\~ao}
\qquad Um algoritmo de ordena\c{c}\~ao \'e um m\'etodo que recebe como entrada uma lista de objetos
compar\'aveis e retorna uma permuta\c{c}\~ao da lista, na qual os objetos est\~ao ordenados. Formalmente,
para um m\'etodo $f : [A] \rightarrow [A]$ ser considerado um algoritmo de ordena\c{c}\~ao correto, ele deve
satisfazer, para qualquer lista $xs$:

\begin{itemize}
  \item $[f(xs)]_{i} \le [f(xs)]_{j}$, sempre que $i \le j$.
  \item $count(x, xs) = count(x, f(xs)) ,  \forall x \in xs$. Nesse caso, $count(y, ys)$ \'e a fun\c{c}\~ao
        que conta o n\'umero de ocorr\^encias do objeto $y$ na lista $ys$.
\end{itemize}

\qquad Os seguintes exemplos mostram sa\'idas corretas, considerando o problema da ordena\c{c}\~ao, para a
entrada indicada:

\begin{itemize}
  \item Entrada: $[9, 7, 10, 2, 4]$, Sa\'ida: $[2, 4, 7, 9, 10]$
  \item Entrada: $[1, 2, 1, 3]$, Sa\'ida: $[1, 1, 2, 3]$
\end{itemize}

\qquad Por outro lado, o seguinte exemplo mostra um resultado incorreto, j\'a que temos duas ocorr\^encias
do n\'umero $2$ na lista de entrada, e apenas uma na lista de sa\'ida.

\begin{itemize}
  \item Entrada: $[2, 2, 4, 1]$, Sa\'ida: $[1, 2, 4]$
\end{itemize}

\qquad Existem diversos algoritmos de ordena\c{c}\~ao conhecidos. Esse trabalho ir\'a se restringir ao
Insertion Sort e o Merge Sort \cite{1}. Essa escolha foi feita por causa de restri\c{c}\~oes de tempo para
o trabalho, e pelo fato de, esses dois serem os \'unicos algoritmos de ordena\c{c}\~ao
formalizados pela biblioteca do Lean, o que os torna os mais interessantes do ponto de vista
da comunidade do assistente.

\section{Complexidade Temporal}

\qquad Se forem analisados como fun\c{c}\~oes matem\'aticas, todos os algoritmos de ordena\c{c}\~ao corretos
s\~ao equivalentes. Contudo, do ponto de vista computacional eles podem se diferenciar no tempo necess\'ario
para resolver o problema da ordena\c{c}\~ao.

\qquad Para que se possa avaliar formalmente a efici\^encia de um algoritmo, deve-se definir uma m\'etrica
que associe a execu\c{c}\~ao dele em uma entrada a um n\'umero natural, tal que este esteja associado \`a
quantidade de instru\c{c}\~oes que um computador teria que executar para realizar aquele algoritmo naquela
entrada. A m\'etrica mais comum para algoritmos de ordena\c{c}\~ao \'e o n\'umero de compara\c{c}\~oes feitas
entre os elementos da lista que ele est\'a ordenando.

\qquad Uma vez que a m\'etrica foi explicitada, deve-se encontrar alguma rela\c{c}\~ao entre o seu valor para
uma entrada espec\'ifica e o tamanho daquela entrada. Por exemplo, no caso dos algoritmos de ordena\c{c}\~ao
iremos comparar o n\'umero de compara\c{c}\~oes executadas com o n\'umero de elementos da lista de entrada.
Ser\'a demonstrado que, para uma lista de tamanho $n$, o Insertion Sort nunca utiliza mais do que $n^{2}$
compara\c{c}\~oes, enquanto o Merge Sort n\~ao passa de $8 n \log_{2} (n)$. Esses dois fatos,
juntos com exemplos em que os algoritmos atingem esses limites, mostram que o Merge Sort \'e mais eficiente
do que o Insertion Sort.

\qquad Na literatura, existem demonstra\c{c}\~oes de que o Merge Sort utiliza menos do que $n\log_{2}(n)$
compara\c{c}\~oes no pior caso \cite{1}, o que \'e um limite mais forte do que o apresentado neste trabalho.
Contudo, na literatura \'e considerada uma implementa\c{c}\~ao que usa um vetor em vez de uma lista,
fazendo com que n\~ao seja necess\'ario usar opera\c{c}\~oes para dividir a lista em duas. No caso do Insertion
Sort, sabe-se que ele nunca faz mais do
que $n(n - 1) / 2$ compara\c{c}\~oes, o que \'e um limite um pouco mais forte
do que o deste trabalho.

\section{Polimorfismo Param\'etrico}

\qquad Existem diversas linguagens de programa\c{c}\~ao que suportam polimorfismo sobre seus tipos. Isso
significa que \'e poss\'ivel definir fun\c{c}\~oes sem fixar concretamente alguns dos tipos sobre a qual elas
agem, fazendo com que se possa us\'a-las em objetos de qualquer tipo. Al\'em disso, algumas linguagens
aceitam que, nessa fun\c{c}\~oes, se restrinja os tipos aceitos \`aqueles que possuam alguma propriedade.
Esta \'e uma varia\c{c}\~ao do polimorfismo, conhecido como polimorfismo param\'etrico.

\qquad Como foi mencionado no in\'icio dessa se\c{c}\~ao, os algoritmos de ordena\c{c}\~ao que estamos
considerando agem sobre listas de objetos de qualquer tipo, desde que seja poss\'ivel comparar elementos
desse tipo. Para expressar essa restri\c{c}\~ao nas linguagens de programa\c{c}\~ao, usa-se o polimorfismo
param\'etrico, restringindo os tipos aceitos \`aqueles que implementam alguma fun\c{c}\~ao que compare
elementos daquele tipo. Para que o algoritmo funcione corretamente, essa fun\c{c}\~ao deve induzir uma ordem
total sobre o tipo. Isso significa que, se a fun\c{c}\~ao de compara\c{c}\~ao sobre um tipo $T$ \'e definida
pelo operador bin\'ario ``$\le$'', as seguintes propriedades devem valer:

\begin{itemize}
  \item $\forall t : T, \quad t \le t$
  \item $\forall t_{1}, t_{2} : T, \quad t_{1} \le t_{2} \wedge t_{2} \le t_{1} \leftrightarrow t_{1} = t_{2}$
  \item $\forall t_{1}, t_{2}, t_{3} : T, \quad t_{1} \le t_{2} \wedge t_{2} \le t_{3} \rightarrow t_{1} \le t_{3}$
\end{itemize}

\qquad Al\'em disso, a fun\c{c}\~ao deve ser total, isto \'e, deve estar definida para todos os pares de
elementos daquele tipo. Sem essa restri\c{c}\~ao, podem existir listas para as quais n\~ao faz sentido
definir se est\~ao ordenadas ou n\~ao, o que torna imposs\'ivel o desenvolvimento de algoritmos para resolver
esse problema.

\section{Indu\c{c}\~ao Transfinita}

\qquad Uma t\'atica muito comum para demonstrar teoremas \'e a indu\c{c}\~ao. Dado um predicado $P(n)$ (que
depende do n\'umero natural $n$), ao se demonstrar a veracidade de $P(0)$, e de $P(k) \Rightarrow P(k + 1)$,
sendo $k$ um n\'umero natural qualquer, o princ\'ipio da indu\c{c}\~ao matem\'atica afirma que $P(n)$ \'e
verdadeiro para todo $n$. Neste trabalho, ser\'a usado amplamente uma vers\~ao generalizada desse
princ\'ipio, conhecida como Indu\c{c}\~ao Transfinita. Nesse caso, deve-se supor que $P(k)$ \'e verdadeiro
para todo $k < n$. Se essa hip\'otese implicar na veracidade de $P(n)$, ent\~ao, pelo princ\'ipio de
Indu\c{c}\~ao Transfinita, obt\^em-se a veracidade de $P(m)$, para todo $m$ natural.

\qquad Eventualmente ser\'a necess\'ario usar a indu\c{c}\~ao com uma hip\'otese da forma $P(f(n))$, para
algum predicado $P$ e fun\c{c}\~ao $f : \mathbb{N} \rightarrow \mathbb{N}$. Isso \'e v\'alido, desde que se
demonstre que $f$ \'e estritamente decrescente. Com essa propriedade, temos a garantia que $f(n) < n$,
logo a hip\'otese de indu\c{c}\~ao se aplica a $f(n)$. Caso exista algum $n$ tal que $f(n) \geq n$,
estar\'iamos usando uma hip\'otese que n\~ao temos para demonstrar o teorema, o que \'e inv\'alido.

\section{Lean}

\qquad Lean \cite{2} \'e um assistente de demonstra\c{c}\~oes. Estes s\~ao linguagens de programa\c{c}\~ao que
possibilitam que seu usu\'ario enuncie e demonstre teoremas, os quais s\~ao verificados pelo compilador do
assistente, que \'e capaz de determinar se a demonstra\c{c}\~ao \'e v\'alida ou n\~ao.

\qquad Assim como muitos outros assistentes, Lean \'e baseado em \textbf{tipos
  dependentes} \cite{typesAtWork}. Ou seja, cada proposi\c{c}\~ao que se enuncia dentro da linguagem \'e
representada internamente por um tipo, o qual depende de valores de outros tipos. Um exemplo para essa
id\'eia \'e a igualdade. Ao enunciar um teorema como $a = b$, sendo $a, b : T$, o s\'imbolo ``$=$'' age como
um construtor de tipos, que toma dois valores do tipo $T$ como par\^ametro e retorna o tipo relativo a
igualdade entre os dois valores. Nesse contexto, construir um termo do tipo ``$a = b$'' equivale a demonstrar
a igualdade entre esses dois valores.

\qquad O processo de demonstrar teoremas usando apenas tipos dependentes se resume, na maior parte das vezes, a
compor fun\c{c}\~oes e analisar casos para conseguir um termo de um determinado tipo, algo que \'e muito dif\'icil
de ser praticado em teoremas que envolvem uma longa argumenta\c{c}\~ao, al\'em de produzir demonstra\c{c}\~oes
pouco leg\'iveis. O modo convencional de realizar essa tarefa \'e muito diferente. Nele,
o matem\'atico pode estruturar a demonstra\c{c}\~ao da forma que preferir, sendo poss\'ivel organizar
ideias complexas de forma menos confusa, o que facilita o trabalho de quem est\'a escrevendo e de quem ir\'a
ler o argumento depois.
Lean fornece abstra\c{c}\~oes ao usu\'ario para que o processo de demonstrar teoremas se torne semelhante ao
convencional. Tais abstra\c{c}\~oes s\~ao conhecidas como t\'aticas. Al\'em de prover as t\'aticas
correspondentes as t\'ecnicas de demonstra\c{c}\~ao mais comuns (como indu\c{c}\~ao, demonstra\c{c}\~ao por
contradi\c{c}\~ao, etc.), o assistente tamb\'em permite que o usu\'ario desenvolva suas pr\'oprias t\'aticas, por
meio de metaprograma\c{c}\~ao \cite{metaprog}.

\qquad Lean tamb\'em apresenta um certo n\'ivel de automatiza\c{c}\~ao de demonstra\c{c}\~oes. Dado um certo fato
que deseja-se demonstrar, \'e poss\'ivel requisitar ao assistente que se pesquise na biblioteca de teoremas j\'a
demonstrados alguma combina\c{c}\~ao que, ou sirva como demonstra\c{c}\~ao direta do que se pretende demonstrar,
ou funcione como uma etapa intermedi\'aria, simplificando o fato a ser demonstrado. Tal funcionalidade \'e
especialmente \'util se usada em conjunto com a biblioteca mantida pela comunidade, conhecida como
\textit{mathlib} \cite{3}. Nela s\~ao mantidas formaliza\c{c}\~oes extensas de diversas \'areas da matem\'atica e da
computa\c{c}\~ao, as quais podem ser usadas livremente para ajudar nas demonstra\c{c}\~oes de teoremas que o
usu\'ario desejar. Em particular, nessa biblioteca existe a formaliza\c{c}\~ao dos algoritmos de ordena\c{c}\~ao,
as quais ser\~ao usadas ao longo deste trabalho. Tais formaliza\c{c}\~oes
n\~ao incluem propriedades relacionadas a complexidade temporal desses algoritmos. Isso n\~ao acontece
apenas nesses dois algoritmos,
esse tipo de propriedade ainda n\~ao foi formalizada para nenhum algoritmo dessa biblioteca.

\qquad O seguinte c\'odigo apresenta a defini\c{c}\~ao dos n\'umeros naturais, da adi\c{c}\~ao sobre eles e a
demonstra\c{c}\~ao da comutatividade dessa opera\c{c}\~ao em Lean, junto com dois lemas necess\'arios para essa
demonstra\c{c}\~ao:

\inputminted{lean}{add.lean}

\chapter{Defini\c{c}\~oes dos Algoritmos}

\qquad Como foi mencionado, nesse trabalho ser\'a abordada a complexidade
temporal de dois algoritmos: Insertion Sort e Merge Sort. Nessa se\c{c}\~ao
eles ser\~ao definidos em linguagem matem\'atica.

\qquad A defini\c{c}\~ao original desses algoritmos (que \'e usada pela \textit{mathlib})
n\~ao registra a quantidade de compara\c{c}\~oes que
s\~ao feitas, o que torna imposs\'ivel que se demonstre
teoremas sobre esses n\'umeros em Lean. Por causa
disso, definimos novas fun\c{c}\~oes que produzem, al\'em de listas,
n\'umeros naturais que determinam o n\'umero de opera\c{c}\~oes executadas.

\qquad Este trabalho ir\'a apresentar demonstra\c{c}\~oes que as listas
produzidas por essas novas fun\c{c}\~oes s\~ao id\^enticas \`as produzidas pelas
fun\c{c}\~oes definidas originalmente, dado que elas receberam a mesma entrada. Al\'em
disso, ser\~ao definidas fun\c{c}\~oes que usam a estrutura da lista para
determinar exatamente quantas compara\c{c}\~oes s\~ao efetivamente realizadas
pelos algoritmos. Com isso, demonstraremos que o n\'umero natural retornado pelo
algoritmo modificado est\'a contando essas compara\c{c}\~oes corretamente. Finalmente,
demonstraremos um limite para esse n\'umero, em fun\c{c}\~ao do tamanho da lista de entrada.
Considerando uma lista de tamanho $n$, mostraremos que o Insertion Sort se limita a $n^{2}$
compara\c{c}\~oes, enquanto o Merge Sort nunca passa de $8 n \log_{2} (n)$.

\qquad Para definir os algoritmos, \'e necess\'ario definir formalmente como se
constroem as listas que ser\~ao usadas. Usaremos o s\'imbolo $[]$ para indicar
uma lista sem nenhum elemento e o s\'imbolo $::$ para indicar a constru\c{c}\~ao
de uma lista, que ter\'a como primeiro elemento a entidade a esquerda do
s\'imbolo, seguido pela lista de elementos a direita dele. Por exemplo, a
constru\c{c}\~ao $3 :: 2 :: 6 :: []$ representa a lista que cont\'em o n\'umero
$3$, seguido por $2$, seguido por $6$.

\section{Insertion Sort}

\qquad A seguir apresentamos a defini\c{c}\~ao original da fun\c{c}\~ao auxiliar do
Insertion Sort, que chamamos de $insert$. Dado um elemento $x$ e uma lista ordenada $xs$,
$insert(x,xs)$ produz uma nova lista, inserindo $x$ em $xs$ de modo que o resultado
permane\c{c}a ordenado. Assim como todas as outras fun\c{c}\~oes que ser\~ao definidas,
sua defini\c{c}\~ao \'e dada por meio de casamento de padr\~oes,
dependendo da forma da lista de entrada. No caso da lista ter mais de um elemento, usamos
a fun\c{c}\~ao recursivamente.

\begin{equation} \label{ins_base}
insert(x, []) = [x]
\end{equation}

\begin{equation} \label{ins_ind}
insert(x, y : ys) =
  \begin{cases}
    x :: y :: ys \text{, \qquad\,\,\,\,\, se } x \le y\\
    y :: insert(x, ys) \text{, caso contr\'ario}
  \end{cases}
\end{equation}

\qquad A vers\~ao estendida funciona exatamente da mesma forma, mas produz tamb\'em um
n\'umero natural, que conta o n\'umero de opera\c{c}\~oes realizadas.

\begin{equation} \label{ins_ext_base}
insertExtended(x, []) = ([x], 0)
\end{equation}

\begin{equation} \label{ins_ext_ind}
insertExtended(x, y :: ys) =
  \begin{cases}
    (x :: y :: ys, 1) \text{, se } x \le y\\
    (y :: l, 1 + n) \text{, caso contr\'ario,}\\ \text{\quad \textbf{onde: }} (l, n) := insertExtended(x, ys)
  \end{cases}
\end{equation}

\qquad Usando $insert$, podemos definir o Insertion Sort.
Essa fun\c{c}\~ao recebe uma lista $xs$ e produz uma permuta\c{c}\~ao ordenada de $xs$.
As equa\c{c}\~oes \ref{ins_sort_base} e \ref{ins_sort_ind} representam sua vers\~ao original, enquanto
\ref{ins_sort_ext_base} e \ref{ins_sort_ext_ind} representam sua vers\~ao estendida.

\begin{equation} \label{ins_sort_base}
insertionSort([]) = []
\end{equation}
\begin{equation} \label{ins_sort_ind}
insertionSort(x :: xs) = insert(x, insertionSort(xs))
\end{equation}

\begin{equation} \label{ins_sort_ext_base}
insertionSortExtended([]) = ([], 0)
\end{equation}


\begin{equation}\label{ins_sort_ext_ind}
\begin{split}
  insertionSortExtended(x :: xs) &= (l', n + m),\\
  \text{\textbf{onde:} }(l, m) &:= insertionSortExtended(xs)\\
  \text{\quad} (l', n) &:= insertExtended(x, l)
\end{split}
\end{equation}

\section{Merge Sort}
\qquad Agora vamos definir o Merge Sort. Para isso, precisaremos de duas fun\c{c}\~oes
auxiliares. A primeira, chamada $split$, recebe uma lista e produz um par de listas,
na qual a primeira cont\'em todos os elementos em posi\c{c}\~oes pares da lista de entrada
e a segunda os elementos de posi\c{c}\~ao \'impar. As equa\c{c}\~oes \ref{split_base} e \ref{split_ind}
representam a defini\c{c}\~ao original e \ref{split_ext_base} e \ref{split_ext_ind} a vers\~ao estendida.
Note que, apesar do $split$ n\~ao executar nenhuma compara\c{c}\~ao, \'e necess\'ario contar quantas vezes
ele \'e usado, pois este n\'umero tem a mesma ordem de grandeza do tamanho da lista, logo essa fun\c{c}\~ao
pode interferir na performance do Merge Sort.

\begin{equation}\label{split_base}
split([]) = ([], [])
\end{equation}
\begin{equation}
\begin{split}\label{split_ind}
  split(x :: xs) &= (x :: l_{2}, l_{1}),\\
  \text{\textbf{onde:} } (l_{1}, l_{2}) &:= split(xs)
\end{split}
\end{equation}

\begin{equation}\label{split_ext_base}
  splitExtended([]) = ([], [], 0)
\end{equation}
\begin{equation}\label{split_ext_ind}
\begin{split}
  splitExtended(x :: xs) &= (x :: l_{2}, l_{1}, 1 + n)\\
  \text{\textbf{onde:} } (l_{1}, l_{2}, n) &:= splitExtended(xs)
\end{split}
\end{equation}


\qquad A segunda fun\c{c}\~ao auxiliar, que chamamos de $merge$, recebe duas listas
ordenadas e produz uma nova lista, intercalando todos os elementos das listas
de entrada, de modo que a lista final esteja ordenada. As equa\c{c}\~oes \ref{merge_1},
\ref{merge_2} e \ref{merge_3} s\~ao sua vers\~ao original, e as equa\c{c}\~oes
\ref{merge_ext_1}, \ref{merge_ext_2} e \ref{merge_ext_3} s\~ao sua vers\~ao estendida.

\begin{equation} \label{merge_1}
merge(xs, []) = xs
\end{equation}
\begin{equation} \label{merge_2}
merge([], ys) = ys
\end{equation}

\begin{equation} \label{merge_3}
merge(x :: xs, y :: ys) =
     \begin{cases}
         x :: merge(xs, y :: ys) \text{,\qquad se $x \le y$} \\
       y :: merge(x :: xs, ys) \text{,\qquad caso contr\'ario}
     \end{cases}
\end{equation}

\begin{equation} \label{merge_ext_1}
mergeExtended(xs, []) = (xs, 0)
\end{equation}
\begin{equation} \label{merge_ext_2}
mergeExtended([], ys) = (ys, 0)
\end{equation}

\begin{equation} \label{merge_ext_3}
mergeExtended(x :: xs, y :: ys) =
     \begin{cases}
       (x :: l_{1}, 1 + n) \text{,\qquad \, se $x \le y$} \\
       (y :: l_{2}, 1 + m) \text{,\qquad caso contr\'ario} \\
       \text{\quad \textbf{onde: }} (l_{1}, n) := mergeExtended(xs, y :: ys) \\
       \text{\qquad \quad \,\,\,\,\, } (l_{2}, m) := mergeExtended(x :: xs, ys)
     \end{cases}
\end{equation}

\qquad Com essas duas fun\c{c}\~oes, podemos finalmente definir o Merge Sort. Assim como o Insertion
Sort, o Merge Sort recebe uma lista $xs$ e produz uma permuta\c{c}\~ao ordenada de $xs$. As
equa\c{c}\~oes \ref{ms_1}, \ref{ms_2} e \ref{ms_3} representam sua defini\c{c}\~ao original,
enquanto as equa\c{c}\~oes \ref{ms_ext_1}, \ref{ms_ext_2} e \ref{ms_ext_3} representam
sua vers\~ao estendida.

\begin{equation}\label{ms_1}
mergeSort([]) = []
\end{equation}

\begin{equation}\label{ms_2}
mergeSort([x]) = [x]
\end{equation}

\begin{equation}\label{ms_3}
\begin{split}
  mergeSort(x_{1} :: x_{2} :: xs) &= merge(l_{1}', l_{2}'),\\
  \text{\textbf{onde:} } (l_{1}, l_{2}) &:= split(x_{1} :: x_{2} :: xs),\\
  l_{1}' &:= mergeSort(l_{1}),\\
  l_{2}' &:= mergeSort(l_{2})
\end{split}
\end{equation}

\begin{equation}\label{ms_ext_1}
mergeSortExtended([]) = ([], 0)
\end{equation}

\begin{equation}\label{ms_ext_2}
mergeSortExtended([x]) = ([x], 0)
\end{equation}

\begin{equation}\label{ms_ext_3}
\begin{split}
  mergeSortExtended(x_{1} :: x_{2} :: xs) &= (l, n + m_{1} + m_{2} + p),\\
  \text{\textbf{onde:} } (l_{1}, l_{2}, n) &:= splitExtended(x_{1} :: x_{2} :: xs),\\
  (l_{1}', m_{1}) &:= mergeSortExtended(l_{1}),\\
  (l_{2}', m_{2}) &:= mergeSortExtended(l_{2}),\\
  (l, p) &:= mergeExtended(l_{1}', l_{2}')
\end{split}
\end{equation}

\section{Fun\c{c}\~oes para Certifica\c{c}\~ao}

\qquad Tamb\'em \'e necess\'ario garantir que o n\'umero que estamos calculando
est\'a capturando corretamente nossa inten\c{c}\~ao, que \'e contar o n\'umero
de opera\c{c}\~oes realizadas por cada algoritmo.  Para isso, tamb\'em
precisamos definir fun\c{c}\~oes que calculam, para uma determinada lista,
quantas opera\c{c}\~oes os algoritmos fazem.

\qquad Sabemos que $insert$ realiza uma compara\c{c}\~ao para cada elemento do prefixo da lista
de entrada que \'e menor do que o elemento que estamos inserindo, e mais uma para o primeiro
que n\~ao \'e menor (caso ele exista). A seguinte fun\c{c}\~ao usa essa informa\c{c}\~ao
para produzir o n\'umero exato de compara\c{c}\~oes:

\begin{equation} \label{comp_ins_base}
  comparisonsInsert(x, []) = 0
\end{equation}


\begin{equation} \label{comp_ins_ind}
\begin{split}
  comparisonsInsert(x, y :: ys) =
    \begin{cases}
      1, &\text{se } x \le y \\
      1 + comparisonsInsert(x, ys), &\text{caso contr\'ario}
    \end{cases}
\end{split}
\end{equation}


\qquad No caso do $merge$, para cada elemento da primeira lista, devemos encontrar o primeiro
elemento da segunda que n\~ao \'e menor do que ele. Ao encontrarmos, repetimos o processo com o pr\'oximo
elemento da primeira lista e o sufixo que sobrou da segunda lista. A seguir, definimos uma
fun\c{c}\~ao auxiliar para contar as opera\c{c}\~oes do $merge$. Ela
recebe um elemento e uma lista ordenada, e remove o prefixo da lista que \'e menor do que o elemento
recebido:

\begin{equation} \label{rem_pref_base}
  removePrefix(x, []) = []
\end{equation}

\begin{equation} \label{rem_pref_ind}
\begin{split}
  removePrefix(x, y :: ys) =
    \begin{cases}
      y :: ys, &\text{se } x \le y \\
      removePrefix(x, ys), &\text{caso contr\'ario}
    \end{cases}
\end{split}
\end{equation}

\qquad Agora vamos definir a fun\c{c}\~ao que calcula o n\'umero de compara\c{c}\~oes em uma execu\c{c}\~ao
do $merge$. Note que, pela similiaridade entre
o $insert$ e as opera\c{c}\~oes que o $merge$ realiza em cada um dos elementos,
podemos usar a fun\c{c}\~ao $comparisonsInsert$ para ajudar a contar esse n\'umero.

\begin{equation} \label{comp_merge_base_1}
  comparisonsMerge(xs, []) = 0
\end{equation}

\begin{equation} \label{comp_merge_base_2}
  comparisonsMerge([], ys) = 0
\end{equation}

\begin{equation} \label{comp_merge_ind}
  \begin{split}
  comparisonsMerge(x :: xs, ys) &= comparisonsInsert(x, ys) + comparisonsMerge(xs, ys')\\
  \text{\textbf{onde: }} ys' &:= removePrefix(x, ys)
  \end{split}
\end{equation}

\chapter{Demonstra\c{c}\~oes Convencionais}

Nesta se\c{c}\~ao iremos apresentar e demonstrar, de forma convencional,
os principais teoremas deste trabalho. Isso \'e importante pois a
argumenta\c{c}\~ao utilizada nessas demonstra\c{o}\~oes \'e essencialmente a
mesma que a do c\'odigo que apresentaremos na se\c{c}\~ao 5, assim, poderemos
focar apenas no c\'odigo em si na pr\'oxima se\c{c}\~ao. Al\'em disso, essa
apresenta\c{c}\~ao ir\'a tornar mais clara as semelhan\c{c}as entre as
demonstra\c{c}\~oes convencionais e o processo de demonstrar teoremas em
assistentes como o Lean. Os fatos que
estamos interessados em formalizar s\~ao os seguintes:
\begin{itemize}
  \item O primeiro elemento do par retornado nas defini\c{c}\~oes estendidas \'e igual ao retorno das defini\c{c}\~oes originais correspondentes.
  \item O segundo elemento do par retornado nas defini\c{c}\~oes estendidas \'e limitado por uma fun\c{c}\~ao (a ser especificada) do tamanho da lista de entrada.
  \item O segundo elemento do par retornado nas defini\c{c}\~oes estendidas \'e igual ao retorno das fun\c{c}\~oes de certifica\c{c}\~ao correspondentes.
\end{itemize}

\qquad Ao longo dos teoremas, usaremos $A$ para indicar um tipo gen\'erico que possui uma ordem total induzida pelo
operador ``$\le$''. Quase todas as demonstra\c{c}\~oes s\~ao feitas por meio de indu\c{c}\~ao na lista de entrada,
portanto, omitiremos essa informa\c{c}\~ao nelas. Sempre usaremos $n$ para indicar $length(l)$, $n_{1}$ para indicar
$length(l_{1})$ e $n_{2}$ para indicar $length(l_{2})$.

\section{Insertion Sort}

\newtheorem{teorema}{Teorema}[section]

\newtheorem{lema}{Lema}[section]

\begin{teorema} \label{insert_eq}
  $\forall a : A, l : [A], \,\, proj_1(insertExtended(a, l)) = insert(a, l)$
\end{teorema}
\noindent \textit{Demonstra\c{c}\~ao.}

  \textbf{Passo Base: } $l = []$.
  \begin{align*}
    proj_{1}(insertExtended(a, [])) &= proj_{1}([a], 0)  &\text{Pela equa\c{c}\~ao \ref{ins_ext_base}} \\
                                      &= [a] &\text{Pela defini\c{c}\~ao de } proj_{1} \\
    &= insert(a, [])  &\text{Pela equa\c{c}\~ao \ref{ins_base}}
  \end{align*}

  \textbf{Passo Indutivo: } $l = h :: t$.

  \textbf{Hip\'otese Indutiva: } $proj_{1}(insertExtended(a, t)) = insert(a, t)$.

  Caso 1: $a \le h$
    \begin{align*}
      proj_{1}(insertExtended(a, h : t)) &= proj_{1}(a :: h :: t, \star) &\text{Pela equa\c{c}\~ao \ref{ins_ext_ind}} \\
      &= a :: h :: t &\text{Pela defini\c{c}\~ao de } proj_1 \\
      &= insert(a, h :: t) &\text{Pela equa\c{c}\~ao \ref{ins_ind}}
    \end{align*}
    \newpage

    Caso 2: $a \not\le h$ \\

      $proj_{1}(insertExtended(a, h :: t))$
    \begin{align*}
       &=  proj_{1}(h :: proj_{1}(insertExtended(a, t)), \star) &\text{Pela equa\c{c}\~ao \ref{ins_ext_ind}}\\
       &= proj_{1}(h :: insert(a, t), \star) &\text{Por HI} \\
       &= h :: insert(a, t) &\text{Pela defini\c{c}\~ao de $proj_{1}$} \\
      &= insert(a, h :: t) &\text{Pela equa\c{c}\~ao \ref{ins_ind}}
    \end{align*}
\qed

\begin{teorema}
  $\forall l : [A], \,\, proj_{1}(insertionSortExtended(l)) = insertionSort(l)$
\end{teorema}

\noindent \textit{Demonstra\c{c}\~ao.}

\textbf{Passo Base: } $l = []$
\begin{align*}
  proj_{1}(insertionSortExtended([])  &= proj_{1}([], 0) &\text{Pela equa\c{c}\~ao \ref{ins_sort_ext_base}} \\
                                                               &= [] &\text{Pela defini\c{c}\~ao de } proj_1 \\
  &= insertionSort([]) &\text{Pela equa\c{c}\~ao \ref{ins_sort_base}}
\end{align*}


\textbf{Passo Indutivo: } $l = h :: t$

\textbf{Hip\'otese Indutiva: } $proj_{1}(insertionSortExtended(a, t)) = insertionSort(a, t)$.\\

Note: definimos $xs := proj_{1}(insertionSortExtended(t))$\\

$proj_{1}(insert\_sort\_extended(h :: t))$
\begin{align*}
  &= proj_{1}(proj_{1}(insertExtended(h, xs)), \star) &\text{Pela equa\c{c}\~ao \ref{ins_sort_ext_ind}} \\
                                          &= proj_{1}(insert(h, xs), \star) &\text{Pelo teorema \ref{insert_eq}} \\
  &= insert(h, xs) &\text{Pela defini\c{c}\~ao de } proj_1 \\
  &= insert(h, proj_{1}(insertionSortExtended(t))) &\text{Pela defini\c{c}\~ao de $xs$}\\
  &= insert(h, insertionSort(t)) &\text{Por HI} \\
  &= insertionSort(h :: t) &\text{Pela equa\c{c}\~ao \ref{ins_sort_ind}}
\end{align*}
\qed


\begin{teorema} \label{count_ins}
  $\forall a : A, l : [A], \,\, proj_{2}(insertExtended(a, l)) \le n$
\end{teorema}

\noindent \textit{Demonstra\c{c}\~ao.}

\textbf{Passo Base: } $l = []$; $n = 0$

\begin{align*}
  proj_{2}(insertExtended(a, [])) &= proj_{2}(\star, 0) &\text{Pela equa\c{c}\~ao \ref{ins_ext_base}} \\
  &= 0 &\text{Pela defini\c{c}\~ao de } proj_{2}
\end{align*}

\textbf{Passo Indutivo: } $l = h :: t$; $n = 1 + n'$

\textbf{Hip\'otese Indutiva: } $proj_{2}(insertExtended(a, t)) \le n'$

Caso 1: $a \le h$
\begin{align*}
  proj_{2}(insertExtended(a, h :: t)) &= proj_{2}(\star, 1) &\text{Pela equa\c{c}\~ao \ref{ins_ext_ind}} \\
  &= 1 &\text{Pela defini\c{c}\~ao de } proj_2
\end{align*}

Como $l$ \'e da forma $h :: t$, ela tem pelo menos um elemento, logo a desigualdade vale.

Caso 2: $a \not\le h$ \\

$proj_{2}(insertExtended(a, h :: t))$
\begin{align*}
   &= proj_{2}(\star, 1 + proj_{2}(insertExtended(a, t))) &\text{Pela equa\c{c}\~ao \ref{ins_ext_ind}} \\
   &= 1 + proj_{2}(insertExtended(a, t)) &\text{Pela defini\c{c}\~ao de } proj_2 \\
   &\le 1 + n' &\text{Por HI} \\
   &= n &\text{Pela defini\c{c}\~ao de n'}
\end{align*}

\qed

\begin{teorema}
  $\forall l : [A], \,\, proj_{2}(insertionSortExtended(l)) \le n^{2}$
\end{teorema}

\noindent \textit{Demonstra\c{c}\~ao.}

\textbf{Passo Base: } $l = []$; $n = 0$
\begin{align*}
  proj_{2}(insertionSortExtended([])) &= proj_{2}(\star, 0) &\text{Pela equa\c{c}\~ao \ref{ins_sort_ext_base}} \\
  &= 0 &\text{Pela defini\c{c}\~ao de } proj_2
\end{align*}

\textbf{Passo Indutivo: } $l = h :: t$; $n = 1 + n'$

\textbf{Hip\'otese Indutiva: } $proj_{2}(insertionSortExtended(t)) \le (n')^{2}$

Note: definimos $m := proj_{2}(insertExtended(h, t))$ \\

$proj_{2}(insertionSortExtended(h :: t))$
\begin{align*}
   &= proj_{2}(\star, proj_{2}(insertionSortExtended(t)) + m) &\text{Pela equa\c{c}\~ao \ref{ins_sort_ext_ind}} \\
                                             &= proj_{2}(insertionSortExtended(t)) + m &\text{Pela defini\c{c}\~ao de } proj_{2} \\
   &\le (n')^{2} + m &\text{Por IH} \\
   &\le (n')^{2} + n' &\text{Pelo teorema \ref{count_ins}} \\
   &=n'(n' + 1) &\text{Por distributividade} \\
   &=n'(n) &\text{Pela defini\c{c}\~ao de $n'$} \\
  &< n^{2} &\text{Pois $n' < n$}
\end{align*}

\qed

\section{Merge Sort}

\begin{teorema}\label{split_eq}
  $\forall l : [A], \,\, proj_{1,2}(splitExtended(l)) = split(l)$
\end{teorema}

\noindent \textit{Demonstra\c{c}\~ao.}

\textbf{Passo Base: } $l = []$

\begin{align*}
  proj_{1,2}(splitExtended([])) &= proj_{1,2}([], [], 0) &\text{Pela equa\c{c}\~ao \ref{split_ext_base}} \\
                                &= ([],[]) &\text{Pela defini\c{c}\~ao de $proj_{1,2}$} \\
  &= split([]) &\text{Pela equa\c{c}\~ao \ref{split_base}}
\end{align*}

\textbf{Passo Indutivo: } $l = h :: t$

\textbf{Hip\'otese Indutiva: } $proj_{1,2}(splitExtended(t)) = split(t)$

Note: definimos $(l_{1}, l_{2}, \star) := splitExtended(t)$

\begin{align*}
  proj_{1,2}(splitExtended(h : t)) &= proj_{1,2}(h :: l_{2}, l_{1}, \star) &\text{Pela equa\c{c}\~ao \ref{split_ext_ind}} \\
                                     &= (h :: l_{2}, l_{1}) &\text{Pela defini\c{c}\~ao de $proj_{1,2}$} \\
  &= split(h :: t') &\text{Pela equa\c{c}\~ao \ref{split_ind}}
\end{align*}

No \'ultimo termo, $t'$ indica alguma lista tal que $split(t') = (l_{1},
l_{2})$. Por hip\'otese indutiva, $t$ satisfaz esse requerimento. Al\'em disso,
$split$ \'e claramente uma fun\c{c}\~ao injetiva. Logo, $t$ \'e a \'unica lista
que satisfaz essa igualdade, e temos $split(h :: t') = split(h :: t)$.

\qed

\begin{teorema}\label{merge_eq}
  $\forall l_{1}, l_{2} : [A], \,\, proj_1(mergeExtended(l_{1}, l_{2})) = merge(l_{1}, l_{2})$
\end{teorema}

\noindent \textit{Demonstra\c{c}\~ao.}

\textbf{Passo Base: } $l_{1} = []$

\begin{align*}
  proj_{1}(mergeExtended([], l_{2})) &= proj_{1}(l_{2}, \star) &\text{Pela equa\c{c}\~ao \ref{merge_ext_2}} \\
  &= l_{2} &\text{Pela defini\c{c}\~ao de } proj_1\\
  &= merge([], l_{2}) &\text{Pela equa\c{c}\~ao \ref{merge_2}}
\end{align*}

\textbf{Passo Indutivo: } $l_{1} = h_{1} :: t_{1}$

\textbf{Hip\'otese Indutiva: } $\forall l_{2} : [A], \,\, proj_{1}(mergeExtended(t_{1}, l_{2})) = merge(t_{1}, l_{2})$

Para demonstrar o passo indutivo, faremos indu\c{c}\~ao na segunda lista:

\textbf{Passo Base 2: }: $l_{2} = []$

\begin{align*}
  proj_{1}(mergeExtended(h_{1} :: t_{1}, [])) &= proj_{1}(h_{1} : t_{1}, \star) &\text{Pela equa\c{c}\~ao \ref{merge_ext_1}} \\
  &= h_{1} :: t_{1} &\text{Pela defini\c{c}\~ao de } proj_1 \\
  &= merge(h_{1} :: t_{1}, []) &\text{Pela equa\c{c}\~ao \ref{merge_1}}
\end{align*}

\textbf{Passo Indutivo 2: } $l_{2} = h_{2} :: t_{2}$

\textbf{Hip\'otese Indutiva 2: } $\forall l_{1} : [A], \,\, proj_{1}(mergeExtended(l_{1}, t_{1})) = merge(l_{1}, t_{1})$

Note: definimos $l'_{1} := proj_{1}(mergeExtended(t_{1}, h_{2} :: t_{2}))$

Caso 1: $h_{1} \le h_{2}$\\

$ proj_{1}(mergeExtended(h_{1} :: t_{1}, h_{2} :: t_{2})) $
\begin{align*}
 &= proj_{1}(h_{1} :: l'_{1}, \star) &\text{Pela equa\c{c}\~ao \ref{merge_ext_3}} \\
                                                          &= h_{1} :: l'_{1} &\text{Pela defini\c{c}\~ao de $proj_{1}$} \\
  &= h_{1} :: merge(t_{1}, h_{2} :: t_{2}) &\text{Por HI} \\
  &= merge(h_{1} :: t_{1}, h_{2} :: t_{2}) &\text{Pela equa\c{c}\~ao \ref{merge_3}}
\end{align*}

Observe que, na terceira igualdade estamos usando a primeira hip\'otese de
indu\c{c}\~ao, tomando $l_{2} = h_{2} :: t_{2}$. A demonstra\c{c}\~ao do segundo caso, em que $h_{1} \not\le h_{2}$, \'e totalmente an\'aloga, usando a segunda hip\'otese de indu\c{c}\~ao.

\qed

Os seguintes lemas ser\~ao \'uteis em alguns dos pr\'oximos teoremas:

\begin{lema} \label{split_le}
  $\forall l : [A], \,\, length(proj_{1}(splitExtended(l))) \le length(l)\, \wedge$

  \qquad \qquad \qquad \qquad \,\, $length(proj_{2}(splitExtended(l))) \le length(l)$
\end{lema}

\noindent \textit{Demonstra\c{c}\~ao.}

\textbf{Passo Base: } $l = []$

\begin{align*}
  length(proj_{1}(splitExtended([]))) &= length(proj_{1}([], \star, \star)) &\text{Pela equa\c{c}\~ao \ref{split_ext_base}}\\
                                        &= length([]) &\text{Pela defini\c{c}\~ao de $proj_{1}$ }\\
  &= 0 &\text{Pela defini\c{c}\~ao de $length$}\\
\end{align*}

A demonstra\c{c}\~ao do segundo item \'e an\'aloga.

\textbf{Passo Indutivo: } $l = h :: t$

\textbf{Hip\'otese Indutiva: } $length(proj_{1}(splitExtended(t))) \le length(t)\, \wedge$

\qquad \qquad \qquad \qquad \quad \,\,\,\,\,\,  $length(proj_{2}(splitExtended(t))) \le length(t)$


Note: definimos $(l_{1}, l_{2}, \star) := splitExtended(t)$


$length(proj_{1}(splitExtended(h :: t)))$
\begin{align*}
  &= length(proj_{1}(h :: l_{2}, \star, \star)) &\text{Pela equa\c{c}\~ao \ref{split_ext_ind}} \\
  &= length(h :: l_{2}) &\text{Pela defini\c{c}\~ao de $proj_{1}$}\\
  &= 1 + length(l_{2}) &\text{Pela defini\c{c}\~ao de $length$}\\
  &\le 1 + length(t) &\text{Por HI}\\
  &= length(h :: t) &\text{Pela defini\c{c}\~ao de $length$}
\end{align*}

A demonstra\c{c}\~ao do segundo item \'e an\'aloga.
\qed

\begin{lema} \label{split_lt}
  $\forall h_{1}, h_{2} : A, t : [T],$

  \qquad $length(proj_{1}(splitExtended(h_{1} :: h_{2} :: t))) < length(h_{1}
    :: h_{2} :: t)$
\end{lema}

\noindent \textit{Demonstra\c{c}\~ao.}

Note: esta \'e a \'unica demonstra\c{c}\~ao que n\~ao usa indu\c{c}\~ao.

Definimos $(l_{1}, l_{2}, \star) := splitExtended(t)$

\begin{align*}
  length(proj_{1}(splitExtended(h_{1} :: h_{2} :: t))) &=
    length(proj_{1}(h_{1} :: l_{1}, \star, \star)) &\text{Pela equa\c{c}\~ao \ref{split_ext_ind}}\\
                                                       &= length(h_{1} :: l_{1}) &\text{Pela defini\c{c}\~ao de $proj_{1}$}\\
                                                       &= 1 + length(l_{1}) &\text{Pela defini\c{c}\~ao de $length$}\\
                                                       &\le 1 + length(t) &\text{Pela lema \ref{split_le}}\\
                                                       &< 2 + length(t)\\
  &= length(h_{1} :: h_{2} :: t) &\text{Pela defini\c{c}\~ao de $length$}
\end{align*}

\qed

\begin{lema}
  $\forall h_{1}, h_{2} : A, t : [T],$

  \qquad $length(proj_{2}(splitExtended(h_{1} :: h_{2} :: t))) < length(h_{1}
    :: h_{2} :: t)$
\end{lema}

\noindent \textit{Demonstra\c{c}\~ao.}

An\'aloga \`a do lema anterior.

\begin{teorema} \label{merge_sort_eq}
  $\forall l : [A], \,\, proj_{1}(mergeSortExtended(l)) = mergeSort(l)$
\end{teorema}

\noindent \textit{Demonstra\c{c}\~ao.}


Caso 1:  $l = []$
\begin{align*}
  proj_{1}(mergeSortExtended([])) &= proj_{1}([], \star) &\text{Pela equa\c{c}\~ao \ref{ms_ext_1}}\\
                                      &= [] &\text{Pela defini\c{c}\~ao de $proj_{1}$}\\
  &= mergeSort([]) &\text{Pela equa\c{c}\~ao \ref{ms_1}}
\end{align*}

Caso 2:  $l = [h]$
\begin{align*}
  proj_{1}(mergeSortExtended([h])) &= proj_{1}([h], \star) &\text{Pela equa\c{c}\~ao \ref{ms_ext_2}}\\
                                      &= [h] &\text{Pela defini\c{c}\~ao de $proj_{1}$}\\
  &= mergeSort([h]) &\text{Pela equa\c{c}\~ao \ref{ms_2}}
\end{align*}

Caso 3: $l = h_{1} :: h_{2} :: t$

Note: definimos
\begin{center}
$(l'_{1}, l'_{2}) := split(l)$\\
$(l_{1}, l_{2}, \star) := splitExtended(l)$\\
$(ls_{1}, \star) := mergeSortExtended(l_{1})$\\
$(ls_{2}, \star) := mergeSortExtended(l_{2})$
\end{center}

Para demonstrar esse caso, ser\'a necess\'ario usar a indu\c{c}\~ao transfinita, apresentada na se\c{c}\~ao 2.
Considere o predicado $P(m)$ sendo $\forall l : [A], \,\, length(l) = m \rightarrow \\ proj_{1}(mergeSortExtended(l)) =
mergeSort(l)$. O teorema \ref{merge_sort_eq} equivale  a $\forall m, P(m)$. Vamos assumir que o teorema vale para toda
lista que tem tamanho menor que $l$. Isso implica que ele vale para $l_{1}$ e $l_{2}$, pois o lema \ref{split_lt} afirma
que a fun\c{c}\~ao $split$ diminui o tamanho de listas que tem pelo menos 2 elementos.\\

$proj_{1}(mergeSortExtended(h_{1} :: h_{2} :: t))$
\begin{align*}
  &= proj_{1}(mergeExtended(ls_{1}, ls_{2})) &\text{Pela equa\c{c}\~ao \ref{ms_ext_3}}\\
  &= merge(ls_{1}, ls_{2}) &\text{Pelo teorema \ref{merge_eq}}\\
  &= merge(mergeSort(l_{1}), mergeSort(l_{2})) &\text{Por HI}\\
  &= merge(mergeSort(l'_{1}), mergeSort(l'_{2})) &\text{Pelo teorema \ref{split_eq}}\\
  &= mergeSort(h_{1} :: h_{2} :: t) &\text{Pela equa\c{c}\~ao \ref{ms_3}}
\end{align*}

\qed

\begin{teorema}\label{split_time}
  $\forall l : [A], \,\, proj_{3}(splitExtended(l)) = n$
\end{teorema}

\noindent \textit{Demonstra\c{c}\~ao.}

\textbf{Passo Base: } $l = []$
\begin{align*}
  proj_{3}(splitExtended([])) &= proj_{3}([], [], 0) &\text{Pela equa\c{c}\~ao \ref{split_ext_base}} \\
  &= 0 &\text{Pela defini\c{c}\~ao de $proj_{3}$}
\end{align*}

\textbf{Passo Indutivo: } $l = h :: t$; $n = 1 + n'$

\textbf{Hip\'otese Indutiva: } $proj_{3}(splitExtended(t)) = n'$

Note: definimos $m := proj_{3}(splitExtended(t))$

\begin{align*}
  proj_{3}(splitExtended(h :: t)) &= proj_{3}(\star, \star, 1 + m) &\text{Pela equa\c{c}\~ao \ref{split_ext_ind}}\\
                                   &= 1 + m &\text{Pela defini\c{c}\~ao de $proj_{3}$}\\
                                   &= 1 + n' &\text{Por HI} \\
  &= n &\text{Pela defini\c{c}\~ao de $n'$}
\end{align*}


\qed

\begin{teorema} \label{merge_time}
  $\forall l_{1}, l_{2} : [A], \,\, proj_{2}(mergeExtended(l_{1}, l_{2})) \le n_{1} + n_{2}$
\end{teorema}

\noindent \textit{Demonstra\c{c}\~ao.}

\textbf{Passo Base: } $l_{1} = []$
\begin{align*}
  proj_{2}(mergeExtended([], l_{2})) &= proj_{2}(\star, 0) &\text{Pela equa\c{c}\~ao \ref{merge_ext_2}}\\
                                       &= 0 &\text{Pela defini\c{c}\~ao de $proj_{2}$} \\
  &\le 0 + n_{2}
\end{align*}

\textbf{Passo Indutivo: } $l_{1} = h_{1} :: t_{1}$; $n_{1} = 1 + n'_{1}$

\textbf{Hip\'otese Indutiva: } $\forall l_{2} : [A], \,\, proj_{2}(mergeExtended(t_{1}, l_{2})) \le n'_{1} + n_{2}$

Mais uma vez, para demonstrar o passo indutivo, usaremos indu\c{c}\~ao na segunda lista.

\textbf{Passo Base 2: } $l_{2} = []$
\begin{align*}
  proj_{2}(mergeExtended(h_{1} :: t_{1}, [])) &= proj_{2}(\star, 0) &\text{Pela equa\c{c}\~ao \ref{merge_ext_1}}\\
                                               &= 0 &\text{Pela defini\c{c}\~ao de $proj_{2}$} \\
  &\le n_{1} + 0
\end{align*}

\textbf{Passo Indutivo 2: } $l_{2} = h_{2} :: t_{2}$; $n_{2} = 1 + n'_{2}$

\textbf{Hip\'otese Indutiva 2: } $\forall l_{1} : [A], \,\, proj_{2}(mergeExtended(l_{1}, t_{2})) \le n_{1} + n'_{2}$

Note: definimos $m_{1} := proj_{2}(mergeExtended(t_{1}, h_{2} :: t_{2}))$

Caso 1: $h_{1} \le h_{2}$

\begin{align*}
  proj_{2}(mergeExtended(h_{1} :: t_{1}, h_{2} :: t_{2})) &= proj_{2}(\star, 1 + m_{1}) &\text{Pela equa\c{c}\~ao \ref{merge_ext_3}} \\
                                                          &= 1 + m_{1} &\text{Pela defini\c{c}\~ao de $proj_{2}$} \\
                                                          &\le 1 + n'_{1} + n_{2} &\text{Por HI} \\
  &= n_{1} + n_{2} &\text{Pela defini\c{c}\~ao de $n'_{1}$}
\end{align*}

Observe que, na terceira igualdade estamos usando a primeira hip\'otese de
indu\c{c}\~ao, tomando $l_{2} = h_{2} :: t_{2}$. A prova do segundo caso, em que $h_{1} \not\le h_{2}$ \'e totalmente an\'aloga, usando a segunda hip\'otese de indu\c{c}\~ao.

\qed

Para demonstrar o pr\'oximo teorema, vamos precisar do seguinte lema:

\begin{lema}\label{split_half}
  $\forall l : [A], \,\, length(proj_{1}(splitExtended(l))) \le \ceil{\frac{n}{2}}\,\, \wedge$

  \qquad \qquad \qquad \qquad \,\, $length(proj_{2}(splitExtended(l))) \le \floor{\frac{n}{2}}$
\end{lema}

\noindent \textit{Demonstra\c{c}\~ao.}

\textbf{Passo Base: } $l = []$

$length(proj_{1}(splitExtended([])))$
\begin{align*}
  &= length(proj_{1}([], \star, \star)) &\text{Pela equa\c{c}\~ao \ref{split_ext_base}}\\
  &= length([]) &\text{Pela defini\c{c}\~ao de $proj_{1}$}\\
  &= 0 &\text{Pela defini\c{c}\~ao de $length$}
\end{align*}

A demonstra\c{c}\~ao da segunda desigualdade \'e an\'aloga.

\textbf{Passo Indutivo: } $l = h :: t$; $n = 1 + n'$

\textbf{Hip\'otese Indutiva: } $length(proj_{1}(splitExtended(t))) \le \ceil{\frac{n'}{2}} \,\, \wedge$

\qquad \qquad \qquad \qquad \quad \,\,\,\,\,\, $length(proj_{2}(splitExtended(t))) \le \floor{\frac{n'}{2}}$

Note: definimos $(l_{1}, l_{2}, \star) := splitExtended(t)$\\

$length(proj_{1}(splitExtended(h :: t)))$
\begin{align*}
  &= length(proj_{1}(h :: l_{2}, \star, \star)) &\text{Pela equa\c{c}\~ao \ref{split_ext_ind}}\\
  &= length(h :: l_{2}) &\text{Pela defini\c{c}\~ao de $proj_{1}$}\\
  &= 1 + length(l_{2}) &\text{Pela defini\c{c}\~ao de $length$}\\
  &\le 1 + \floor*{\frac{n'}{2}} &\text{Por HI}\\
  &= \floor*{\frac{n' + 2}{2}}\\
  &= \floor*{\frac{n + 1}{2}} &\text{Pela defini\c{c}\~ao de $n'$}\\
  &= \ceil*{\frac{n}{2}}
\end{align*}

A demonstra\c{c}\~ao da segunda desigualdade \'e an\'aloga.

\qed

\begin{teorema}
  $\forall h : A, t : [A], \, proj_{2}(mergeSortExtended(h :: t)) \le 8 n \log_{2}n$
\end{teorema}
(nesse caso, $n$ = $lenght(h :: t)$)\\

\noindent \textit{Demonstra\c{c}\~ao.}

Caso 1: $t = []$
\begin{align*}
  proj_{2}(mergeSortExtended([h])) &= proj_{2}(\star, 0) &\text{Pela equa\c{c}\~ao \ref{ms_ext_2}}\\
                                      &= 0 &\text{Pela defini\c{c}\~ao de $proj_{2}$}\\
  &\le 8 * 1 * log_{2}(1)
\end{align*}

Caso 2: $t = h' :: t'$

Usaremos o mesmo argumento do teorema \ref{merge_sort_eq} para usar a hip\'otese
indutiva em $proj_{1}(splitExtended(h :: h' :: t'))$ e
$proj_{2}(splitExtended(h :: h' :: t'))$.

Note: definimos
\begin{center}
  $(l_{1}, l_{2}, p) := splitExtended(h :: h' :: t')$\\
  $m_{1} := length(l_{1})$\\
  $m_{2} := length(l_{2})$\\
  $e_{1} := proj_{2}(mergeSortExtended(l_{1}))$\\
  $e_{2} := proj_{2}(mergeSortExtended(l_{2}))$\\
  $q := proj_{2}(mergeExtended(l_{1}, l_{2}))$\\
\end{center}

$proj_{2}(mergeSortExtended(h :: h' :: t'))$
\begin{align*}
  &= proj_{2}(\star, q + m_{1} + m_{2} + p) &\text{Pela equa\c{c}\~ao \ref{ms_ext_3}}\\
  &= q + e_{1} + e_{2} + p &\text{Pela defini\c{c}\~ao de $proj_{2}$}\\
  &= q + e_{1} + e_{2} + n &\text{Pelo teorema \ref{split_time}}\\
  &\le m_{1} + m_{2} + e_{1} + e_{2} + n &\text{Pelo teorema \ref{merge_time}}\\
  &= n + e_{1} + e_{2} + n\\
  &\le 2n + 8m_{1}\log_{2}(m_{1}) + 8m_{2}\log_{2}(m_{2}) &\text{Por HI}\\
  &\le 2n + 8\ceil*{\frac{n}{2}}\log_{2}(\ceil*{\frac{n}{2}}) + 8\floor*{\frac{n}{2}}\log_{2}(\floor*{\frac{n}{2}}) &\text{Pelo lema \ref{split_half}}\\
  &\le 2n + 8(\frac{n + 1}{2})\log_{2}(\frac{n + 1}{2}) + 8(\frac{n}{2})\log_{2}(\frac{n}{2}) \\
  &\le 2n + 4n\log_{2}(n) + 4\log_{2}(n) + 4n(\log_{2}(n) - 1)\\
  &\le 2n + 4n\log_{2}(n) + 2n + 4n(\log_{2}(n) - 1)\\
  &= 4n\log_{2}(n) + 4n\log_{2}(n) - 4n + 4n\\
  &= 8n\log_{2}(n)
\end{align*}

Observe que na d\'ecima linha estamos usando $4\log_{2}(n) \le 2n$. Al\'em disso, na quarta linha estamos usando o fato da soma dos tamanhos das duas listas geradas por $split$ ser igual ao tamanho da lista original, o que \'e simples de ser verificado usando indu\c{c}\~ao.

\qed

\section{Certifica\c{c}\~oes}

\begin{teorema}
  $\forall a : A, l : [A], \,\, proj_{2}(insertExtended(l)) = comparisonsInsert(l)$
\end{teorema}

\noindent \textit{Demonstra\c{c}\~ao.}

\textbf{Passo Base: } $l = []$

\begin{align*}
  proj_{2}(insertExtended(a, [])) &= proj_{2}(\star, 0) &\text{Pela equa\c{c}\~ao \ref{ins_ext_base}}\\
                                 &= 0 &\text{Pela defini\c{c}\~ao de } proj_{2} \\
  &= comparisonsInsert(a, []) &\text{Pela equa\c{c}\~ao \ref{comp_ins_base}}
\end{align*}

\textbf{Passo Indutivo: } $l = h : t$

\textbf{Hip\'otese Indutiva: } $proj_{2}(insertExtended(a, t)) = comparisonsInsert(a, t)$

Note: definimos $m := proj_{2}(insertExtended(a, t))$.

Caso 1: $a \le h$
\begin{align*}
  proj_{2}(insertExtended(a, h :: t)) &= proj_{2}(\star, 1) &\text{Pela equa\c{c}\~ao \ref{ins_ext_ind}}\\
                                       &= 1 &\text{Pela defini\c{c}\~ao de } proj_{2} \\
  &= comparisonsInsert(a, h :: t) &\text{Pela equa\c{c}\~ao \ref{comp_ins_ind}}
\end{align*}

Caso 2: $a \not\le h$ \\

$proj_{2}(insertExtended(a, h :: t))$
\begin{align*}
  &= proj_{2}(\star, 1 + m) &\text{Pela equa\c{c}\~ao \ref{ins_ext_ind}} \\
                                    &= 1 + m &\text{Pela defini\c{c}\~ao de } proj_{2} \\
                                    &= 1 + proj_{2}(insertExtended(a, t)) &\text{Pela defini\c{c}\~ao de } m\\
                                    &= 1 + comparisonsInsert(a, t) &\text{Por HI} \\
  &= comparisonsInsert(a, h :: t) &\text{Pela equa\c{c}\~ao \ref{comp_ins_ind}}
\end{align*}

\qed

\begin{teorema}
  $\forall l_{1}, l_{2} : [A], \,\, proj_{2}(mergeExtended(l_{1}, l_{2})) = comparisonsMerge(l_{1}, l_{2})$
\end{teorema}

\noindent \textit{Demonstra\c{c}\~ao.}

\textbf{Passo Base: } $l_{1} = []$\\

$proj_{2}(mergeExtended([], l_{2}))$
\begin{align*}
  &= proj_{2}(\star, 0) &\text{Pela equa\c{c}\~ao \ref{merge_ext_2}} \\
                                       &= 0 &\text{Pela defini\c{c}\~ao de $proj_{2}$} \\
  &= comparisonsMerge([], l_{2}) &\text{Pela equa\c{c}\~ao \ref{comp_merge_base_2}} \\
\end{align*}

\textbf{Passo Indutivo: } $l_{1} = h_{1} :: t_{1}$

\textbf{Hip\'otese Indutiva: }


\qquad $\forall l_{2} : [A], \,\, proj_{2}(mergeExtended(t_{1}, l_{2})) = comparisonsMerge(t_{1}, l_{2})$ \\

Para realizar o passo indutivo faremos indu\c{c}\~ao em $l_{2}$.

\textbf{Passo Base 2: } $l_{2} = []$\\

$proj_{2}(mergeExtended(h_{1} :: t_{1}, []))$
\begin{align*}
   &= proj_{2}(\star, 0) &\text{Pela equa\c{c}\~ao \ref{merge_ext_1}} \\
                                               &= 0 &\text{Pela defini\c{c}\~ao de $proj_{2}$} \\
  &= comparisonsMerge(h_{1} :: t_{1}, []) &\text{Pela equa\c{c}\~ao \ref{comp_merge_base_1}}
\end{align*}

\textbf{Passo Indutivo 2: } $l_{2} = h_{2} :: t_{2}$; $n_{2} = 1 + n'_{2}$

\textbf{Hip\'otese Indutiva 2: }

\qquad $\forall l_{1} : [A], \,\, proj_{2}(mergeExtended(l_{1}, t_{2})) = comparisonsMerge(l_{1}, t_{2})$ \\

Note: definimos
\begin{center}
$m_{1} := proj_{2}(mergeExtended(t_{1}, h_{2} :: t_{2}))$\\
$m_{2} := proj_{2}(mergeExtended(h_{1} :: t_{1}, t_{2}))$\\
$r_{1} := removePrefix(h_{1}, h_{2} :: t_{2})$\\
$r_{2} := removePrefix(h_{1}, t_{2})$
\end{center}

Caso 1: $h_{1} \le h_{2}$\\

$proj_{2}(mergeExtended(h_{1} :: t_{1}, h_{2} :: t_{2}))$
\begin{align*}
   &= proj_{2}(\star, 1 + m_{1}) &\text{Pela equa\c{c}\~ao \ref{merge_ext_3}}\\
  &= 1 + m_{1} &\text{Pela defini\c{c}\~ao de $proj_{2}$} \\
   &= 1 + comparisonsMerge(t_{1}, h_{2} :: t_{2}) &\text{Por HI} \\
  &= comparisonsMerge(h_{1} :: t_{1}, h_{2} :: t_{2}) &\text{Pela equa\c{c}\~ao \ref{comp_merge_ind}}
\end{align*}

Na terceira igualdade estamos usando a primeira hip\'otese indutiva com $l_{2} =
h_{2} :: t_{2}$. Como $h_{1} \le h_{2}$, temos $comparisonsInsert(h_{1}, h_{2}
:: t_{2}) = 1$ e $removePrefix(h_{1}, h_{2} :: t_{2}) = h_{2} :: t_{2}$, logo a \'ultima
igualdade vale. \\


Caso 2: $h_{1} \not \le h_{2}$\\

$proj_{2}(merge\_exteded(h_{1} :: t_{1}, h_{2} :: t_{2}))$
\begin{align*}
  &= proj_{2}(\star, 1 + m_{2}) &\text{Pela equa\c{c}\~ao \ref{merge_ext_3}}\\
  &= 1 + m_{2} &\text{Pela defini\c{c}\~ao de $proj_{2}$}\\
  &= 1 + comparisonsMerge(h_{1} :: t_{1}, t_{2}) &\text{Por HI} \\
  &= 1 + comparisonsInsert(h_{1}, t_{2}) + comparisonsMerge(t_{1}, r_{2}) &\text{Pela equa\c{c}\~ao \ref{comp_merge_ind}}\\
  &= comparisonsInsert(h_{1}, h_{2} :: t_{2}) + comparisonsMerge(t_{1}, r_{2}) &\text{Pela equa\c{c}\~ao \ref{comp_ins_ind}}\\
  &= comparisonsInsert(h_{1}, h_{2} :: t_{2}) + comparisonsMerge(t_{1}, r_{1}) &\text{Pois $h_{1} \not \le h_{2}$}\\
  &= comparisonsMerge(h_{1} :: t_{1}, h_{2} :: t_{2}) &\text{Pela equa\c{c}\~ao \ref{comp_merge_ind}}
\end{align*}

Na terceira igualdade estamos usando a segunda hip\'otese indutiva com $l_{1} =
h_{1} :: t_{1}$. Note que,
como $h_{1} \not \le h_{2}$, vale que $comparisonsInsert(h_{1}, h_{2} ::
t_{2}) = 1 + \\comparisonsInsert(h_{1}, t_{2})$. Pelo mesmo motivo,
$removePrefix(h_{1}, h_{2} :: t_{2}) =\\removePrefix(h_{1}, t_{2})$.

\qed


\chapter{Formaliza\c{c}\~oes em Lean}

Nessa se\c{c}\~ao iremos definir as fun\c{c}\~oes da se\c{c}\~ao 3 usando Lean. Al\'em disso, apresentaremos o c\'odigo do enunciado dos principais teoremas da se\c{c}\~ao 4 em Lean, assim como as principais t\'aticas usadas para demonstr\'a-los. O argumento de todas as demonstra\c{c}\~oes em Lean \'e essencialmente o mesmo da se\c{c}\~ao 4.

As seguintes declara\c{c}\~oes informam ao compilador do Lean que usaremos $\alpha$ como um tipo gen\'erico ao longo das defini\c{c}\~oes, e que \textsf{r} \'e uma rela\c{c}\~ao entre elementos de $\alpha$. A segunda linha est\'a definindo o operador \textsf{$\preceq$} como um alias para a rela\c{c}\~ao \textsf{r}. A senten\c{c}a\\ \textsf{[decidable\_rel r]} \'e usada para restringir \textsf{r} \`{a}s rela\c{c}\~oes que s\~ao decid\'iveis, isto \'e, para quaisquer \textsf{a, b : $\alpha$},
\textsf{a $\preceq$ b} nunca \'e indefinido, sendo sempre verdadeiro ou falso. Isto \'e
necess\'ario pois, como foi argumentado, \textsf{r} deve sempre ser uma rela\c{c}\~ao de
ordem total. Uma poss\'ivel melhoria seria implementar as outras restri\c{c}\~oes que limitariam \textsf{r} para apenas rela\c{c}\~oes de ordem total. Tais restri\c{c}\~oes n\~ao s\~ao implementadas na \textit{mathlib}.

\inputminted{lean}{declarations.lean}

\section{Fun\c{c}\~oes}

\subsection{Insertion Sort}

As seguintes defini\c{c}\~oes correspondem a implementa\c{c}\~ao das duas vers\~oes do
Insert e Insertion Sort em Lean (equa\c{c}\~oes~\ref{ins_base}
at\'e~\ref{ins_sort_ext_ind}). Na primeira linha de cada uma, temos o nome da
fun\c{c}\~ao, seguido pelo seu tipo. Nas linhas seguintes temos sua defini\c{c}\~ao
definida por meio do casamento de padr\~oes com a lista de entrada.

\inputminted{lean}{insertion_sort.lean}
\inputminted{lean}{insertion_sort_modified.lean}

\subsection{Merge Sort}

As pr\'oximas defini\c{c}\~oes correspondem a implementa\c{c}\~ao das duas
vers\~oes do Merge Sort e suas fun\c{c}\~oes auxiliares em Lean
(equa\c{c}\~oes~\ref{merge_1} at\'e~\ref{ms_ext_3}).


Nas duas vers\~oes do Merge Sort foi necess\'ario usar o modo de t\'aticas do Lean.
Esse modo, usualmente usado para demonstrar teoremas, permite que o usu\'ario use
algumas express\~oes especiais para ajudar na defini\c{c}\~ao que est\'a sendo
feita. No caso do Merge Sort, ele foi usado pois fazemos recurs\~ao na lista gerada
pela fun\c{c}\~ao split. Nesse caso, \'e necess\'ario fornecer ao compilador uma
demonstra\c{c}\~ao que essa recurs\~ao ir\'a, eventualmente, terminar. Para isso,
antes de definir o que ser\'a retornado, usamos o lema \textsf{length\_split\_lt} (que afirma que o tamanho da lista produzida por \textsf{split} \'e menor do que o tamanho da lista original), juntamente com o comando \textsf{using\_well\_founded}, para indicar ao compilador que a m\'etrica usada para determinar o fim da recurs\~ao \'e o tamanho da lista.

\inputminted{lean}{merge_sort.lean}
\inputminted{lean}{merge_sort_modified.lean}

\subsection{Certifica\c{c}\~ao}

As seguintes fun\c{c}\~oes s\~ao as vers\~oes em Lean correspondentes \`as
equa\c{c}\~oes usadas para certifica\c{c}\~ao (equa\c{c}\~oes~\ref{comp_ins_base} at\'e~\ref{comp_merge_ind})

\inputminted{lean}{ins_comp.lean}
\inputminted{lean}{remove_pref.lean}
\inputminted{lean}{merge_comp.lean}

\section{Teoremas}

\subsection{Insertion Sort}

\inputminted{lean}{ins_complexity.lean}

Note: \textsf{.snd} equivale a fun\c{c}\~ao $proj_{2}$ que estamos usando.

Equivalente ao teorema~\ref{count_ins}. A t\'atica \textsf{induction} \'e usada
 para iniciar uma demonstra\c{c}\~ao por indu\c{c}\~ao. No passo indutivo,
usa-se a t\'atica
\textsf{split\_ifs} para dividir entre os casos \textsf{$a \le h$} e \textsf{$a
\not \le h$}. Por causa do \textsf{let} usado na defini\c{c}\~ao do
\textsf{insertExtended}, \'e necess\'ario usar a t\'atica \textsf{cases},
que troca a chamada recursiva \textsf{insertExtended t} pelo par \textsf{(l',
  n)}, seguido pela t\'atica \textsf{unfold}, que reescreve a defini\c{c}\~ao de
\textsf{insertExtended}. A t\'atica \textsf{linarith} \'e usada quando \'e necess\'ario apenas realizar c\'alculos para completar a demonstra\c{c}\~ao.

\subsection{Merge Sort}

\subsection{Certifica\c{c}\~ao}

\chapter{Conclus\~ao}

% \addappheadtotoc
\appendixpage
\clearpage
\appendix
\chapter{Teoremas Secund\'arios da Se\c{c}\~ao 4}

\postextual


% \bibliographystyle{abbrv}
\bibliography{ref}


\end{document}
